\documentclass[12pt,a4paper]{article}
\usepackage[utf8]{inputenc}
\usepackage[croatian]{babel}
\usepackage{graphicx}
\usepackage{hyperref}
\usepackage{listings}
\usepackage{xcolor}
\usepackage{float}
\usepackage{booktabs}
\usepackage{geometry}

\geometry{
    a4paper,
    left=3cm,
    right=3cm,
    top=3cm,
    bottom=3cm
}

% Code listing settings
\lstset{
    basicstyle=\ttfamily\small,
    keywordstyle=\color{blue},
    commentstyle=\color{green!60!black},
    stringstyle=\color{red},
    showstringspaces=false,
    breaklines=true,
    frame=single,
    numbers=left,
    numberstyle=\tiny\color{gray}
}

\hypersetup{
    colorlinks=true,
    linkcolor=blue,
    filecolor=magenta,      
    urlcolor=cyan,
    citecolor=blue
}

\title{
    \Huge\textbf{PlantTracker} \\
    \Large Aplikacija za praćenje i pomoć u održavanju rasta biljaka \\
    \vspace{0.5cm}
    \large Aktivne i temporalne baze podataka u PostgreSQL-u
}

\author{
    [Ime i prezime studenta] \\
    JMBAG: [JMBAG] \\
    \texttt{[email@student.unizg.hr]} \\
    \vspace{0.5cm}
    Fakultet organizacije i informatike \\
    Sveučilište u Zagrebu
}

\date{\today}

\begin{document}

\maketitle
\newpage

\tableofcontents
\newpage

% ============================================
% 1. OPIS APLIKACIJSKE DOMENE
% ============================================

\section{Opis aplikacijske domene}

\subsection{Uvod u domenu}

Aplikacija PlantTracker namijenjena je ljubiteljima biljaka, vrtlarima i svima koji žele sistematski pratiti rast i održavanje svojih biljaka. Moderna tehnologija omogućava nam da digitaliziramo proces brige o biljkama, što rezultira boljim zdravljem biljaka i učinkovitijim upravljanjem vremenom.

\subsection{Koncepti i entiteti domene}

Aplikacijska domena obuhvaća sljedeće ključne koncepte:

\begin{itemize}
    \item \textbf{Biljke} - Centralni entitet sustava. Svaka biljka ima osnovne informacije (naziv, vrsta, lokacija, datum sadnje) te trenutni status zdravlja.
    
    \item \textbf{Događaji} - Sve akcije vezane uz biljku: zalijevanje, gnojenje, presađivanje, rezanje, bolesti i napomene. Događaji čine povijesni zapis aktivnosti.
    
    \item \textbf{Podsjetnici} - Automatski sustav koji generira upozorenja za planirana održavanja (npr. zalijevanje svakih 7 dana).
    
    \item \textbf{Temporalni podaci} - Povijest promjena statusa biljke kroz vrijeme, omogućava uvid u evoluciju biljke.
    
    \item \textbf{Mjerenja rasta} - Kvantitativni podaci o rastu (visina, širina, broj listova) kroz vrijeme.
    
    \item \textbf{Slike} - Vizualna dokumentacija rasta i stanja biljke.
    
    \item \textbf{Notifikacije} - Automatski generirane poruke sustava (aktivne baze podataka).
\end{itemize}

\subsection{Specifičnosti domene}

\begin{enumerate}
    \item \textbf{Temporalna priroda podataka} - Praćenje biljaka inherentno zahtijeva vremensku dimenziju. Potrebno je znati ne samo trenutno stanje, već i povijest promjena statusa, frekvenciju događaja i trendove rasta.
    
    \item \textbf{Potreba za automatizacijom} - Korisnici često zaborave na aktivnosti održavanja. Sustav mora proaktivno podsjećati na potrebne akcije.
    
    \item \textbf{Varijabilnost ciklusa održavanja} - Različite biljke zahtijevaju različite režime održavanja. Sukulente se zalijevaju rijetko, dok tropske biljke trebaju češće zalijevanje.
    
    \item \textbf{Analitika i trendovi} - Korisnici žele vidjeti kako njihove biljke rastu kroz vrijeme i identificirati probleme.
\end{enumerate}

\subsection{Motivacija za PostgreSQL s aktivnim i temporalnim bazama}

\subsubsection{Zašto PostgreSQL?}

PostgreSQL je odabran kao RDBMS iz sljedećih razloga:

\begin{itemize}
    \item \textbf{Napredni temporalni tipovi podataka} - PostgreSQL nudi \texttt{TIMESTAMP}, \texttt{DATE}, \texttt{INTERVAL}, \texttt{TSRANGE} tipove koji omogućuju prirodno modeliranje vremenskih podataka.
    
    \item \textbf{Moćni okidači (triggers)} - Implementacija aktivnih baza podataka kroz sofisticirane okidače koji omogućuju automatizaciju poslovne logike.
    
    \item \textbf{Stored procedure i funkcije} - PL/pgSQL omogućava kompleksnu logiku na razini baze podataka.
    
    \item \textbf{ACID svojstva} - Garantirana konzistentnost podataka što je ključno za integritet temporalnih podataka.
    
    \item \textbf{Napredni indeksi} - GiST indeksi za temporalno pretraživanje.
    
    \item \textbf{Open source} - Besplatna i otvorena tehnologija.
\end{itemize}

\subsubsection{Zašto ne druge tehnologije?}

\begin{table}[H]
\centering
\begin{tabular}{@{}lll@{}}
\toprule
\textbf{Tehnologija} & \textbf{Prednosti} & \textbf{Nedostaci za ovu domenu} \\ \midrule
MongoDB & Fleksibilnost sheme & Slaba podrška za temporalne upite \\
 & Skalabilnost & Nema native triggers \\
 & & Slabija konzistentnost \\ \midrule
Neo4J & Graf relacije & Nije optimiziran za temporalne podatke \\
 & Pattern matching & Kompleksnost za ovaj use case \\ \midrule
MySQL & Široka upotreba & Slabija podrška za temporalne tipove \\
 & Performanse & Manje napredni triggers \\ \bottomrule
\end{tabular}
\caption{Usporedba tehnologija}
\label{tab:tech-comparison}
\end{table}

PostgreSQL je optimalan izbor jer nativno podržava i temporalne podatke i aktivne baze kroz triggers, što su dva ključna zahtjeva aplikacije.

% ============================================
% 2. TEORIJSKI UVOD
% ============================================

\section{Teorijski uvod}

\subsection{Aktivne baze podataka}

\subsubsection{Definicija}

Aktivne baze podataka proširuju tradicionalne baze podataka mogućnošću autonomnog reagiranja na promjene podataka ili vremenske uvjete. Paradigma aktivnih baza zasniva se na ECA (Event-Condition-Action) pravilima:

\begin{itemize}
    \item \textbf{Event} - Događaj koji pokreće pravilo (INSERT, UPDATE, DELETE)
    \item \textbf{Condition} - Uvjet koji mora biti zadovoljen
    \item \textbf{Action} - Akcija koja se izvršava
\end{itemize}

\subsubsection{Implementacija u PostgreSQL-u}

PostgreSQL implementira aktivne baze kroz \textbf{okidače (triggers)}:

\begin{lstlisting}[language=SQL, caption=Primjer okidača]
CREATE OR REPLACE FUNCTION update_reminder_after_event()
RETURNS TRIGGER AS $$
BEGIN
    -- Event: INSERT na events tablicu
    -- Condition: Provjera tipa događaja
    -- Action: Ažuriranje podsjetnika
    UPDATE reminders
    SET last_performed = NEW.event_date,
        next_due = NEW.event_date + frequency_interval
    WHERE plant_id = NEW.plant_id
      AND reminder_type = NEW.event_type;
    RETURN NEW;
END;
$$ LANGUAGE plpgsql;
\end{lstlisting}

\subsubsection{Prednosti aktivnih baza}

\begin{enumerate}
    \item \textbf{Automatizacija poslovne logike} - Smanjuje količinu aplikacijskog koda
    \item \textbf{Konzistentnost} - Osigurava da se pravila primjenjuju bez obzira na izvor promjene
    \item \textbf{Performanse} - Izvršavanje na strani baze često je brže
    \item \textbf{Centralizacija logike} - Lakše održavanje
\end{enumerate}

\subsubsection{Nedostaci aktivnih baza}

\begin{enumerate}
    \item \textbf{Kompleksnost debugiranja} - Teže pratiti tok izvršavanja
    \item \textbf{Rizik od kaskadnih okidača} - Okidač može pokrenuti drugi okidač
    \item \textbf{Vendor lock-in} - Sintaksa specifična za DBMS
    \item \textbf{Skalabilnost} - Mogu usporiti operacije nad podacima
\end{enumerate}

\subsection{Temporalne baze podataka}

\subsubsection{Definicija}

Temporalne baze podataka podržavaju pohranu i upitivanje podataka s vremenskom dimenzijom. Postoje dva glavna koncepta vremena:

\begin{itemize}
    \item \textbf{Valid time} - Vrijeme kada je podatak bio istinit u stvarnom svijetu
    \item \textbf{Transaction time} - Vrijeme kada je podatak zabilježen u bazi
\end{itemize}

\subsubsection{Implementacija u PlantTracker aplikaciji}

Aplikacija koristi \textbf{valid time} pristup za praćenje povijesti statusa biljaka:

\begin{lstlisting}[language=SQL, caption=Temporalna tablica]
CREATE TABLE plant_status_history (
    history_id SERIAL PRIMARY KEY,
    plant_id UUID REFERENCES plants(plant_id),
    status plant_status NOT NULL,
    valid_from TIMESTAMP NOT NULL,
    valid_to TIMESTAMP DEFAULT NULL,
    CONSTRAINT valid_period_check 
        CHECK (valid_to IS NULL OR valid_to > valid_from)
);
\end{lstlisting}

Period \texttt{[valid\_from, valid\_to)} predstavlja vremenski interval valjanosti statusa.

\subsubsection{Temporalni upiti}

PostgreSQL omogućava sofisticirane temporalne upite:

\begin{lstlisting}[language=SQL, caption=Dohvaćanje statusa u određenom trenutku]
SELECT status 
FROM plant_status_history
WHERE plant_id = '...'
  AND valid_from <= '2024-06-15'
  AND (valid_to IS NULL OR valid_to > '2024-06-15')
\end{lstlisting}

\subsubsection{Prednosti temporalnih baza}

\begin{enumerate}
    \item \textbf{Audit trail} - Potpuna povijest promjena
    \item \textbf{Point-in-time queries} - Stanje u bilo kojem trenutku
    \item \textbf{Analiza trendova} - Razumijevanje evolucije podataka
    \item \textbf{Regulatorna usklađenost} - Zadovoljavanje zakonskih zahtjeva
\end{enumerate}

\subsubsection{Nedostaci temporalnih baza}

\begin{enumerate}
    \item \textbf{Veličina baze} - Povijest zauzima prostor
    \item \textbf{Kompleksnost upita} - Temporalni upiti mogu biti kompleksni
    \item \textbf{Performanse} - Potrebni dodatni indeksi
\end{enumerate}

\subsection{Integracija aktivnih i temporalnih baza}

PlantTracker kombinira oba pristupa:

\begin{itemize}
    \item \textbf{Aktivne baze} - Automatsko evidentiranje promjena u temporalnu povijest
    \item \textbf{Temporalne baze} - Pohranjivanje i upitivanje povijesnih podataka
\end{itemize}

Okidači automatski kreiraju nove zapise u temporalnoj tablici kada se promijeni status biljke, čime se osigurava potpuna povijest bez potrebe za ručnim evidentiranjem.

% Nastavlja se u sljedećim sekcijama...

\section{Model baze podataka}

\subsection{Konceptualni model (ERA dijagram)}

% Ovdje će biti ubačen ERA dijagram
% Korisnik može generirati dijagram koristeći draw.io ili pgModeler

\subsection{Opis tablica}

\subsubsection{Tablica plants}

Osnovna tablica koja pohranjuje informacije o biljkama.

\begin{table}[H]
\centering
\small
\begin{tabular}{@{}llll@{}}
\toprule
\textbf{Atribut} & \textbf{Tip} & \textbf{Opis} & \textbf{Ograničenja} \\ \midrule
plant\_id & UUID & Jedinstveni identifikator & PK \\
common\_name & VARCHAR(100) & Uobičajeni naziv & NOT NULL \\
scientific\_name & VARCHAR(150) & Znanstveni naziv & \\
variety & VARCHAR(100) & Varijanta & \\
location & VARCHAR(200) & Lokacija & \\
planting\_date & DATE & Datum sadnje & NOT NULL \\
current\_status & plant\_status & Trenutni status & ENUM \\
created\_at & TIMESTAMP & Datum kreiranja & DEFAULT NOW() \\
updated\_at & TIMESTAMP & Datum ažuriranja & DEFAULT NOW() \\ \bottomrule
\end{tabular}
\caption{Struktura tablice plants}
\end{table}

\subsubsection{Tablica plant\_status\_history}

Temporalna tablica za praćenje povijesti statusa.

\begin{table}[H]
\centering
\small
\begin{tabular}{@{}llll@{}}
\toprule
\textbf{Atribut} & \textbf{Tip} & \textbf{Opis} & \textbf{Ograničenja} \\ \midrule
history\_id & SERIAL & Jedinstveni identifikator & PK \\
plant\_id & UUID & Referenca na biljku & FK \\
status & plant\_status & Status biljke & NOT NULL \\
valid\_from & TIMESTAMP & Početak valjanosti & NOT NULL \\
valid\_to & TIMESTAMP & Kraj valjanosti & CHECK \\
changed\_by & VARCHAR(100) & Tko je promijenio & \\
notes & TEXT & Napomene & \\ \bottomrule
\end{tabular}
\caption{Struktura tablice plant\_status\_history}
\end{table}

\textit{Napomena: Ostale tablice dokumentirane su na sličan način u potpunoj verziji dokumentacije.}

\subsection{Relacije između entiteta}

\begin{itemize}
    \item \textbf{plants 1:N events} - Jedna biljka može imati više događaja
    \item \textbf{plants 1:N reminders} - Jedna biljka može imati više podsjetnika
    \item \textbf{plants 1:N plant\_status\_history} - Jedna biljka ima povijest statusa
    \item \textbf{plants 1:N growth\_measurements} - Jedna biljka ima mjerenja rasta
    \item \textbf{plants 1:N images} - Jedna biljka može imati više slika
    \item \textbf{reminders 1:N notifications} - Jedan podsjetnik može generirati više notifikacija
\end{itemize}

\section{Implementacija}

\subsection{Kreiranje baze podataka}

Instalacija je automatizirana kroz \texttt{install.sh} skriptu:

\begin{lstlisting}[language=bash, caption=Instalacijska skripta]
#!/bin/bash
createdb planttracker
psql -d planttracker -f database/01_schema.sql
psql -d planttracker -f database/02_triggers.sql
psql -d planttracker -f database/03_functions.sql
psql -d planttracker -f database/04_sample_data.sql
\end{lstlisting}

\subsection{Ključni okidači (Triggers)}

\subsubsection{Okidač za praćenje promjena statusa}

\begin{lstlisting}[language=SQL, caption=Trigger za temporalnu povijest]
CREATE TRIGGER trigger_track_status_change
    AFTER UPDATE OF current_status ON plants
    FOR EACH ROW
    WHEN (OLD.current_status IS DISTINCT FROM NEW.current_status)
    EXECUTE FUNCTION track_plant_status_change();
\end{lstlisting}

Ovaj okidač automatski evidentira svaku promjenu statusa u temporalnu tablicu.

\subsubsection{Okidač za automatsko ažuriranje podsjetnika}

\begin{lstlisting}[language=SQL, caption=Trigger za podsjetnika]
CREATE TRIGGER trigger_update_reminder
    AFTER INSERT ON events
    FOR EACH ROW
    EXECUTE FUNCTION update_reminder_after_event();
\end{lstlisting}

Kada se doda novi događaj (npr. zalijevanje), ovaj okidač automatski ažurira povezani podsjetnik i računa sljedeći datum.

\subsection{Ključne funkcije}

\subsubsection{Temporalni upiti}

\begin{lstlisting}[language=SQL, caption=Funkcija za dohvaćanje statusa u trenutku]
CREATE OR REPLACE FUNCTION get_plant_status_at(
    p_plant_id UUID,
    p_timestamp TIMESTAMP
)
RETURNS plant_status AS $$
DECLARE
    result plant_status;
BEGIN
    SELECT status INTO result
    FROM plant_status_history
    WHERE plant_id = p_plant_id
      AND valid_from <= p_timestamp
      AND (valid_to IS NULL OR valid_to > p_timestamp)
    ORDER BY valid_from DESC
    LIMIT 1;
    
    RETURN result;
END;
$$ LANGUAGE plpgsql;
\end{lstlisting}

\subsection{Aplikacijski sloj}

Aplikacija je implementirana u Python Flask frameworku:

\begin{lstlisting}[language=Python, caption=Flask aplikacija]
from flask import Flask, render_template
from database import Database

app = Flask(__name__)
db = Database()

@app.route('/')
def index():
    plants = db.get_plants_overview()
    stats = db.get_dashboard_stats()
    return render_template('index.html', 
                         plants=plants, 
                         stats=stats)

@app.route('/plant/<plant_id>')
def plant_detail(plant_id):
    plant = db.get_plant(plant_id)
    report = db.get_plant_report(plant_id)
    return render_template('plant_detail.html',
                         plant=plant,
                         report=report)
\end{lstlisting}

\section{Primjeri korištenja}

\subsection{Dodavanje nove biljke}

\begin{lstlisting}[language=SQL, caption=Dodavanje biljke]
INSERT INTO plants (common_name, scientific_name, location, planting_date)
VALUES ('Monstera', 'Monstera deliciosa', 'Dnevna soba', '2024-01-15');
\end{lstlisting}

\subsection{Evidentiranje zalijevanja}

\begin{lstlisting}[language=SQL, caption=Dodavanje događaja zalijevanja]
INSERT INTO events (plant_id, event_type, description, amount)
VALUES ('...', 'zalijevanje', 'Redovno zalijevanje', '300ml');
\end{lstlisting}

Ovaj INSERT automatski:
\begin{enumerate}
    \item Pokreće okidač \texttt{trigger\_update\_reminder}
    \item Ažurira povezani podsjetnik
    \item Računa sljedeći datum zalijevanja
    \item Označava postojeće notifikacije kao pročitane
\end{enumerate}

\subsection{Temporalni upiti - status u prošlosti}

\begin{lstlisting}[language=SQL, caption=Upit za status 1. lipnja 2024]
SELECT get_plant_status_at(
    'a0eebc99-9c0b-4ef8-bb6d-6bb9bd380a11', 
    '2024-06-01'::TIMESTAMP
);
\end{lstlisting}

\subsection{Analiza rasta}

\begin{lstlisting}[language=SQL, caption=Trend rasta u zadnjih 90 dana]
SELECT * FROM get_growth_trend(
    'a0eebc99-9c0b-4ef8-bb6d-6bb9bd380a11',
    90
);
\end{lstlisting}

\section{Zaključak}

\subsection{Procjena tehnologije}

PostgreSQL s podrškom za aktivne i temporalne baze podataka pokazao se kao odličan izbor za aplikaciju PlantTracker. Ključne prednosti:

\begin{enumerate}
    \item \textbf{Nativna podrška za temporalne tipove} - Omogućava prirodno modeliranje vremenskih podataka
    \item \textbf{Moćni okidači} - Automatizacija poslovne logike značajno smanjuje kompleksnost aplikacijskog koda
    \item \textbf{Performanse} - GiST indeksi omogućuju brze temporalne upite
    \item \textbf{Konzistentnost} - ACID svojstva garantiraju integritet podataka
\end{enumerate}

\subsection{Ograničenja implementacije}

\begin{enumerate}
    \item \textbf{Kompleksnost debugiranja} - Okidači mogu činiti tok podataka netransparentnim
    \item \textbf{Vendor lock-in} - PL/pgSQL je specifičan za PostgreSQL
    \item \textbf{Skalabilnost} - Za vrlo velike količine povijesnih podataka potrebne su strategije arhiviranja
    \item \textbf{Grafičko sučelje} - Trenutna implementacija ima osnovno web sučelje
\end{enumerate}

\subsection{Mogućnosti proširenja}

\begin{enumerate}
    \item \textbf{PostGIS integracija} - Geografske lokacije biljaka u vrtu
    \item \textbf{Napredna analitika} - Prediktivni modeli za potrebe zalijevanja
    \item \textbf{Mobilna aplikacija} - Native iOS/Android aplikacija
    \item \textbf{IoT integracija} - Senzori vlage tla
    \item \textbf{Multi-tenant arhitektura} - Podrška za više korisnika
\end{enumerate}

\subsection{Zaključna riječ}

PlantTracker uspješno demonstrira primjenu aktivnih i temporalnih baza podataka u praktičnoj domeni. Kombinacija ovih tehnologija omogućava:
\begin{itemize}
    \item Potpunu povijest rasta biljaka
    \item Automatsko podsjećanje na aktivnosti održavanja
    \item Analizu trendova i identifikaciju problema
    \item Jednostavno korištenje kroz intuitivno sučelje
\end{itemize}

Projekt pokazuje da PostgreSQL nije samo relacijska baza podataka, već moćna platforma za implementaciju kompleksnih poslovnih zahtjeva kroz aktivne i temporalne značajke.

\begin{thebibliography}{99}

\bibitem{postgres_docs}
PostgreSQL Global Development Group. (2024). 
\textit{PostgreSQL 16 Documentation}.
\url{https://www.postgresql.org/docs/16/}

\bibitem{temporal_databases}
Snodgrass, R. T. (1999). 
\textit{Developing Time-Oriented Database Applications in SQL}.
Morgan Kaufmann Publishers.

\bibitem{active_databases}
Paton, N. W., \& Díaz, O. (1999). 
\textit{Active database systems}.
ACM Computing Surveys, 31(1), 63-103.

\bibitem{triggers}
PostgreSQL Global Development Group. (2024).
\textit{PostgreSQL Triggers}.
\url{https://www.postgresql.org/docs/16/triggers.html}

\bibitem{temporal_sql}
Kulkarni, K., \& Michels, J. E. (2012).
\textit{Temporal features in SQL: 2011}.
ACM SIGMOD Record, 41(3), 34-43.

\bibitem{flask}
Flask Development Team. (2024).
\textit{Flask Documentation}.
\url{https://flask.palletsprojects.com/}

\end{thebibliography}

\end{document}
